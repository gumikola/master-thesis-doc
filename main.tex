\documentclass[printmode]{mgr}
%opcje klasy dokumentu mgr.cls zostały opisane w dołączonej instrukcji

%poniżej deklaracje użycia pakietów, usunąć to co jest niepotrzebne
\usepackage{polski}
\usepackage[utf8]{inputenc}

%pakiety do grafiki
\usepackage{graphicx}
\usepackage{subfigure}
\usepackage{psfrag}

%pakiety dodające dużo dodatkowych poleceń matematycznych
\usepackage{amsmath}
\usepackage{amsfonts}

%pakiety wspomagające i poprawiające składanie tabel
\usepackage{supertabular}
\usepackage{array}
\usepackage{tabularx}
\usepackage{hhline}

%pakiet wypisujący na marginesie etykiety równań i rysunków
%zdefiniowanych przez \label{}, chcąc wygenerować finalną wersję
%dokumentu wystarczy usunąć poniższą linię
\usepackage{showlabels}

%dane do złożenia strony tytułowej
\title{Optymalizacja procesu produkcji urządzeń elektronicznych}
\engtitle{Optimization of electronic devices manufacturing process}
\author{Mikołaj Guz}
\supervisor{dr hab. Mieczysław Wodecki}

\date{2019} %standardowo u dołu strony tytułowej umieszczany jest bieżący rok, to polecenie pozwala wstawić dowolny rok

%poniżej jest lista kierunków i specjalności na wydziale elektroniki,
%należy wybrać właściwe lub dopisać jeśli nie ma odpowiednich
\field{Automatyka i Robotyka (AIR)}
\specialisation{Technologie informacyjne w systemach automatyki (ART)}


%tutaj zaczyna się właściwa treść dokumentu
\begin{document}

\maketitle %polecenie generujące stronę tytułową 

\tableofcontents %spis treści

\chapter{Wstęp}

\section{Opis procesu}

\paragraph{}
Proces badany w pracy to linia montująca płyty drukowane do finalnych produktów. Sam proces składa się z kilku etapów:
\begin{enumerate}
	\item Pobranie potrzebnych elementów elektronicznych z magazynu.
	\item Uzbrojenie pick\&place w wymagane elementy.
	\item Pobranie gotowych płyt PCB (ang. Printed Circuit Board).
	\item Ręczne nałożenie pasty lutowniczej za pomocą sitodruku
	\item Uruchomienie procesu montażu elementów na maszynie pick\&place
	\item (Opcjonalnie) Umieszczenie ręczne elementów
	\item Przeprowadzenie procesu lutowania w piecu.
	\item (Opcjonalnie) Lutowanie ręczne
	\item (Opcjonalnie) Kontrola gotowej płyty PCB
\end{enumerate}

\paragraph{}
Ograniczenia procesu:
\begin{itemize}
	\item Wszystkie procedury muszą być wykonane w ustalonej kolejności
	\item Dla płyt dwustronnych należy powtórzyć sekwencję od punktu 4 do punktu 7
	\item Pick\&place może obsługiwać tylko jedną płytę PCB
	\item Piec lutowniczy może lutować kilka płyt drukowanych (ilość jest zależna od rozmiaru płyt)
\end{itemize}

\includegraphics{chapters/graph.png}

\addcontentsline{toc}{chapter}{\bibname} %utworzenie w spisietreści pozycji Literatura

\bibliography{bibliography} % wstawia bibliografię korzystając z pliku
\bibliographystyle{plabbrv} %tylko gdy używamy BibTeXa, ustawia polski styl bibliografii

%opcjonalnie może się tu pojawić spis rysunków i tabel
% \listoffigures
% \listoftables
\end{document}
