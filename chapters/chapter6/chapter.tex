\chapter{Podsumowanie}

Celem pracy magisterskiej było stworzenie narzędzia do optymalizacji procesu produkcji urządzeń elektronicznych. Problem jaki należało rozwiązać był problem przepływowy z przezbrojeniem maszyn.

W pierwszym etapie przeanalizowano cały proces produkcji. Wyznaczono diagram przepływu pracy(Rysunek~\ref{DiagFlow}) dzięki czemu można było wyznaczyć ścieżkę krytyczną. Następnie przeanalizowań każde stanowisko pracy w celu znalezienia zależności procesowych oraz ocenić jak można zoptymalizować pracę systemu.

Kolejnym etapem prac było sformułowanie problemu. Dzięki stworzonemu modelowi można było zobaczyć co jest wąskim gardłem w procesie produkcji oraz jaki algorytm użyć w celu optymalizacji.

W celu badań stworzono aplikację w C++. Użyto to tego bibliotek i narzędzi w Qt. Dzięki temu wyborowi można w przyszłości dodać nowe funkcjonalności do programu np. interfejs graficzny. Wykorzystano również w projekcie bazę danych SQLite. Umożliwiło to w sposób spójny dostarczyć potrzebnych danych o poszczególnych projektach do aplikacji.

Następnie wybrano dwa algorytmy do rozwiązania problemu przepływowego z przezbrojeniami: przegląd zupełny oraz symulowane wyżarzanie. Przegląd zupełny to algorytm dokładny. Oznacza to że rozwiązanie otrzymane jest zawsze optymalne lecz jak dowiodły badania okupione duża złożonością obliczeniową. Umożliwiło to wyznaczenie permutacji zadań optymalnej w celu pórwania z rozwiązaniami przybliżonymi. Symulowane wyżarzanie to algorytm z grupy metaheurystycznych. Znajduję on rozwiązanie przybliżone w dość krótkim czasie.

Z wyników badań wynikło że SA jest algorytmem szybki i dającym dość dobre wyniki. Jego wadą jest że jest silnie sparametryzowany więc należy dużą uwagę poświecić na odpowiednie ustawanie parametrów. Kolejnym wnioskiem jest że przegląd zupełny nie nadaje się do dużych zbiorów zadań ($n>11$). Czas wykonania wtedy drastycznie się zwiększa przez co jego przydatność jest mała.

Jak można zauważyć optymalizacja procesu rzeczywistego jest dość trudnym zadaniem. W pracy jest omawiany przypadek z dużą ilością uproszczeń oraz małej ilości maszyn. W warunkach przemysłowych produkcji elektroniki gdzie maszyn jest kilkadziesiąt i nie można sobie pozwolić na jakiekolwiek uproszczenia problem robi się bardziej złożony przez co nalezy użyć bardziej wyrafinowanych technik optymalizacji.