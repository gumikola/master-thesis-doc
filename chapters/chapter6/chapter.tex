\chapter{Podsumowanie}

Celem pracy magisterskiej było wykonanie oprogramowania do optymalizacji procesu produkcji elementów elektronicznych. Proces ten może być zamodelowany jako problem przepływowy z przezbrojeniami maszyn.

W pierwszym etapie przeanalizowano cały proces produkcji. Wyznaczono diagram przepływu elementów (Rysunek~\ref{DiagFlow}), dzięki czemu można było wyznaczyć ścieżkę krytyczną. Następnie przeanalizowano każde stanowisko pracy w celu znalezienia zależności procesowych oraz oceniono w jaki sposób można zoptymalizować pracę systemu.

Kolejnym etapem było sformułowanie problemu. Zbudowany model pozwolił na zaobserwowanie, co jest wąskim gardłem w procesie produkcji oraz jaki algorytm można zastosować do optymalizacji procesu.

Algorytmy (dokładny i przybliżony) zaimplementowano w języku C++. Użyto także bibliotek i narzędzi w Qt. Umożliwia to w przyszłości dodawanie nowych funkcjonalności do programu, np.\ interfejsu graficznego. W projekcie wykorzystano również bazę danych SQLite. Dzięki temu w sposób spójny można dostarczyć do aplikacji potrzebne dane o poszczególnych projektach.

Następnie wybrano dwa algorytmy do rozwiązania problemu przepływowego z przezbrojeniami: przegląd zupełny oraz symulowane wyżarzanie. Przegląd zupełny jest algorytmem dokładnym. Oznacza to, że otrzymane rozwiązanie jest zawsze optymalne, lecz jak dowiodły badania, okupione dużym czasem obliczeń. Stwarza to jednak możliwość wyznaczenia rozwiązania optymalnego, w celu porównania z rozwiązaniami przybliżonymi. Symulowane wyżarzanie jest algorytmem z grupy metaheurystyk. Wyznacza on rozwiązanie przybliżone w dość krótkim czasie.

Na podstawie eksperymentów obliczeniowych można stwierdzić, że SA jest algorytmem szybkim i dającym dość dobre wyniki. Jego wadą jest silnie sparametryzowanie, więc należy dużą uwagę poświęcić na odpowiednie dobranie parametrów. Kolejnym wnioskiem jest to, że przegląd zupełny nie nadaje się do dużych zbiorów zadań ($n>11$). Czas wykonania drastycznie się wtedy zwiększa, przez co jego przydatność jest mała.

Jak można zauważyć, optymalizacja procesu rzeczywistego jest dość trudnym zadaniem. W pracy omawiany jest przypadek z pewnymi uproszczeniami oraz małą liczbą maszyn. W warunkach przemysłowych produkcji elektroniki, gdzie maszyn jest kilkadziesiąt, problem staje się dużo bardziej złożony. W tej sytuacji nie można pozwolić na jakiekolwiek uproszczenia, przez co należy użyć bardziej wyrafinowanych technik optymalizacji.