\chapter{Podsumowanie}

Celem pracy magisterskiej było stworzenie narzędzia do optymalizacji procesu produkcji urządzeń elektronicznych. Zagadnieniem, jakie należało rozwiązać, był problem przepływowy z przezbrojeniem maszyn.

W pierwszym etapie przeanalizowano cały proces produkcji. Wyznaczono diagram przepływu pracy (Rysunek~\ref{DiagFlow}), dzięki czemu można było wyznaczyć ścieżkę krytyczną. Następnie przeanalizowano każde stanowisko pracy w celu znalezienia zależności procesowych oraz oceniono w jaki sposób można zoptymalizować pracę systemu.

Kolejnym etapem było sformułowanie problemu. Stworzony model pozwolił na zaobserwowanie, co jest wąskim gardłem w procesie produkcji oraz jakiego algorytmu należy użyć w celu optymalizacji.

W celu badań stworzono aplikację w C++. Użyto do tego bibliotek i narzędzi w Qt. Umożliwia to w przyszłości dodawanie nowych funkcjonalności do programu, np. interfejsu graficznego. W projekcie wykorzystano również bazę danych SQLite. Dzięki temu w sposób spójny można dostarczyć do aplikacji potrzebne dane o poszczególnych projektach.

Następnie wybrano dwa algorytmy do rozwiązania problemu przepływowego z przezbrojeniami: przegląd zupełny oraz symulowane wyżarzanie. Przegląd zupełny to algorytm dokładny. Oznacza to, że otrzymane rozwiązanie jest zawsze optymalne, lecz jak dowiodły badania, okupione dużą złożonością obliczeniową. Stwarza to możliwość wyznaczenia optymalnej permutacji zadań, w celu porównania z rozwiązaniami przybliżonymi. Symulowane wyżarzanie to algorytm z grupy metaheurystycznych. Znajduje on rozwiązanie przybliżone w dość krótkim czasie.

Z badań wynika, że SA jest algorytmem szybkim i dającym dość dobre wyniki. Jego wadą jest silnie sparametryzowanie, więc należy dużą uwagę poświęcić na odpowiednie dobranie parametrów. Kolejnym wnioskiem jest, że przegląd zupełny nie nadaje się do dużych zbiorów zadań ($n>11$). Czas wykonania drastycznie się wtedy zwiększa, przez co jego przydatność jest mała.

Jak można zauważyć, optymalizacja procesu rzeczywistego jest dość trudnym zadaniem. W pracy omawiany jest przypadek z dużą ilością uproszczeń oraz małą ilością maszyn. W warunkach przemysłowych produkcji elektroniki, gdzie maszyn jest kilkadziesiąt, problem staje się bardziej złożony. W tej sytuacji nie można pozwolić na jakiekolwiek uproszczenia, przez co należy użyć bardziej wyrafinowanych technik optymalizacji.