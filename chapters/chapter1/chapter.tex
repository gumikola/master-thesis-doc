\chapter{Wstęp}

(Tu się będzie znajdować jakiś opis wstępny)

W czasach coraz większej cyfryzacji życia potrzebna jest ogromna ilość różnego typu urządzeń elektronicznych.
W celu dostarczenia niezbędnych ilości należy spróbować usprawnić cały łańcuch produkcji oraz projektowania zaczynając od koncepcji urządzenia do dostarczenia gotowego produktu do użytkownika końcowego.

\section{Cel i założenia pracy}
Celem pracy magisterskiej jest stworzenie narzędzia do optymalizacji procesu produkcji urządzeń elektronicznych.
Aspektem badawczym pracy jest stworzenie modelu procesu produkcji oraz na jego podstawie opracowanie algorytmów optymalizacyjnych
Aspekt inżynierski to zaprojektowanie odpowiednich algorytmów optymalizacyjnych oraz implantacja w oprogramowaniu.

Główne założenia pracy magisterskiej:
\begin{itemize}
    \item opis procesu produkcji,
    \item skonstruowanie modelu matematycznego dla podanego problemu,
    \item implantacja algorytmów rozwiązujące dane zagadnienie,
    \item wykonanie badań,
    \item porównanie wyników.
\end{itemize}
