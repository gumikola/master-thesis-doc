\chapter{Wstęp}

W czasach coraz większej cyfryzacji życia potrzebna jest ogromna ilość różnego typu urządzeń elektronicznych.
W celu dostarczenia niezbędnych ilości należy spróbować usprawnić cały łańcuch produkcji elektroniki. Proces ten jest złożony i potrzeba odpowiednich narzędzi żeby obniżyć czas produkcji, zminimalizować przestoje oraz podnieść jakość elementów końcowych. Jednym z sposobów na to jest wyznaczanie optymalnych harmonogramów zadań co jest kluczowym zadaniem w niniejsze pracy magisterskiej.

W tym dokumencie spróbujemy przeanalizować przykład produkcji elektroniki mało seryjnej oraz znaleźć rozwiązanie dla problemu przepływowego z przezbrojeniami.

\section{Cel i zakres pracy}
Celem pracy magisterskiej było stworzenie narzędzia do optymalizacji procesu produkcji urządzeń elektronicznych.
Aspektem badawczym pracy jest stworzenie modelu procesu produkcji oraz opracowanie algorytmów optymalizacyjnych na jego podstawie.
Aspekt inżynierski to zaprojektowanie odpowiednich algorytmów optymalizacyjnych oraz ich implementacja w oprogramowaniu. Aplikacja wykorzystana pracy zostanie stworzona w języku C++ za pomocą bibliotek Qt.

\breakparagraph{}
Główne założenia pracy magisterskiej:
\begin{itemize}
	\item opis procesu produkcji,
	\item skonstruowanie modelu matematycznego dla danego problemu,
	\item implementacja algorytmów rozwiązujących dane zagadnienie,
	\item wykonanie badań,
	\item porównanie wyników.
\end{itemize}
