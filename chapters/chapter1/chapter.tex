\chapter{Wstęp}

(Tu się będzie znajdować jakiś opis wstępny)

W czasach coraz większej cyfryzacji życia potrzebna jest ogromna ilość różnego typu urządzeń elektronicznych.
W celu dostarczenia niezbędnych ilości należy spróbować usprawnić cały łańcuch produkcji oraz projektowania zaczynając od koncepcji urządzenia do dostarczenia gotowego produktu do użytkownika końcowego.

(wprowadzić pojęcia typu PCB,SMD,a priori,fifo,porządek lekso)

\section{Cel i zakres pracy}
Celem pracy magisterskiej jest stworzenie narzędzia do optymalizacji procesu produkcji urządzeń elektronicznych.
Aspektem badawczym pracy jest stworzenie modelu procesu produkcji oraz opracowanie algorytmów optymalizacyjnych na jego podstawie.
Aspekt inżynierski to zaprojektowanie odpowiednich algorytmów optymalizacyjnych oraz ich implementacja w oprogramowaniu. (tu coś dopisać o oprogramowaniu ale jeszcze nie wiem co)

\breakparagraph{}
Główne założenia pracy magisterskiej:
\begin{itemize}
    \item opis procesu produkcji,
    \item skonstruowanie modelu matematycznego dla danego problemu,
    \item implementacja algorytmów rozwiązujących dane zagadnienie,
    \item wykonanie badań,
    \item porównanie wyników.
\end{itemize}
