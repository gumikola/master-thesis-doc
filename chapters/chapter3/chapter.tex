\chapter{Sformułowanie problemu}
W tym rozdziale opiszemy przebieg rozpatrywanego procesu oraz przedstawimy jego model matematyczny.

% \section{Pare słów o Flow Shop}
% % ogarnij kolejność flow shop gdzie ma być
% \breakparagraph{}
% \textbf{Flow shop ($F_{m}$)} --- model $m$ pracujących maszyn i $n$ zadań. Każde zadaniem musi zostać obsłużone przez każdą maszynę. Kolejność operacji zadania jest ustalona z góry i nie ma możliwości jej zmiany. Dodatkowo zadanie nie może być przetwarzane przez kolejną maszynę, jeżeli nie zakończyło pracy na poprzedniej.


\section{Opis problemu}
Rozpatrywany problem jest to pewne uogólnienie klasycznego w teorii szeregowania permutacyjnego problemu przepływowego (ang. Flow-Shop). Dokładny opis tego problemu oraz algorytm metaheurystyczny jego rozwiązania są przedstawione między innymi w pracach ``Grabowski.J, Wodecki.M''~\cite{GRABOWSKI20041891} oraz ``Nowicki.E, Smutnicki.C''~\cite{NOWICKI1996160}

Dodatkowo w procesie występują przezbrojenia maszyn.

\breakparagraph{}
Ze względu na dużą złożoność procesu założone zostały pewne uproszczenia:
\begin{itemize}
	\item łańcuch transportu pomiędzy maszynami a magazynem został pominięty --- zakładamy, że potrzebne elementy znajdują przy każdym stanowisku,
	\item pomijamy ilość operatorów --- w pracy nie badamy wpływu ilości operatorów na czas trwania procesu, każda maszyna posiada własnego operatora,
	\item piec lutuje jednocześnie tylko jedną płytę PCB --- w warunkach rzeczywistych liczba lutowanych płyt PCB jest ograniczona powierzchnią roboczą pieca. Pominięcie tego założenia wymaga stosowania magazynów pomiędzy maszynami.0
\end{itemize}

\breakparagraph{}
Proces wytwarzania płyt jest realizowany na 6 stanowiskach (maszynach):
\begin{description}
	\item[Maszyna 1] Stanowisko z sitodrukiem manualnym.
	\item[Maszyna 2] Automat pick and place.
	\item[Maszyna 3] Ręczne układanie elementów.
	\item[Maszyna 4] Piec lutowniczy.
	\item[Maszyna 5] Stacja do lutowania ręcznego.
	\item[Maszyna 6] Kontrola wizyjna gotowych elementów.
\end{description}

\newpage{}
W procesie na 2 maszynach jest wykonywane przezbrojenie pomiędzy kolejnymi operacjami:
\begin{enumerate}
	\item Stanowisko sitodruku.
	\item Automat pick\&place
\end{enumerate}

\section{Model matematyczny procesu}

\breakparagraph{}
Zbiór maszyn
\begin{equation}
	M=\lbrace 1, 2, \dots, m \rbrace,
\end{equation}

gdzie $m$ to liczba maszyn. Maszyna reprezentuje stanowisko.

\breakparagraph{}
Zbiór zadań:
\begin{equation}
	J=\lbrace 1, 2, \dots, n \rbrace,
\end{equation}

gdzie $n$ to liczba zadań. Zadaniami są poszczególne płytki, które należy wykonać.

\breakparagraph{}
Zadanie $j \in J$ jest ciągiem $m$ operacji:
\begin{equation}
	O_{1, j}, O_{2, j}, \dots, O_{m, j},
\end{equation}

Operacja $O_{i,j}$ reprezentuje czynności, które należy wykonać na i-tej płytce na j-tym stanowisku.

\breakparagraph{}
Zmienne w modelu:
\begin{itemize}
	\item $p_{k, j}$ --- czas trwania operacji zadania $j$ na maszynie $k$
	\item $S_{k, j}$ --- moment rozpoczęcia operacji zadania $j$ na maszynie $k$
	\item $C_{k, j}$ --- moment zakończenia operacji zadania $j$ na maszynie $k$
\end{itemize}

\breakparagraph{}
Czas przezbrojenia pomiędzy zadaniem $i$ oraz $j$ na $k$-tej maszynie oznacza $s_{i, j}^{k}$, przez $( i \neq j \ oraz\ i, j \in J )$.

\breakparagraph{}
Łatwo zauważyć, że dowolne rozwiązanie problemu może być reprezentowane przez permutację zadania. Przez $\pi$ oznaczamy zbiór wszystkich permutacji elementów zbioru $J$.

Problem polega na wyznaczeniu minimalnego czasu zakończenia procesu. Muszą być przy tym spełnione następujące ograniczenia:
\begin{itemize}
	\item musi być zachowany porządek technologiczny operacji:
		  \begin{equation}
			  O_{1, j} \rightarrow O_{2, j} \rightarrow \dots \rightarrow O_{m, j}
		  \end{equation}
	\item każda operacja może być wykonywana tylko przez jedną maszynę,
	\item żadna maszyna nie może wykonywać więcej niż jednej operacji (uwagi: przypadek kiedy w piecu lutowniczym lutujemy jedną płytę PCB),
	\item wykonywanie żadnej operacji nie może zostać przerwane przed jej zakończeniem,
	\item momenty rozpoczęcia operacji oraz czas przezbrojenia nie mogą być ujemne:
		  \begin{equation}
			  S_{k, \pi(i)}, s^{k}_{\pi(i), \pi(j)} > 0
		  \end{equation}
\end{itemize}

Wprowadźmy następujące oznaczenia:
\begin{itemize}
	\item
	moment rozpoczęcia operacji $O_{i,\pi(j)}$:
	\begin{equation}
		S_{k, \pi(j)}=\max\{C_{k-1, \pi(j)}, C_{k, \pi(j-1)}+s^k_{\pi(j-1), \pi(j)}\},
	\end{equation}
	\item
	moment zakończenia operacji $O_{i,\pi(j)}$:
	\begin{equation}
		C_{k, \pi(j)} = S_{k, \pi(j)} + p_{k, \pi(j)},
	\end{equation}
	gdzie $ j \in J$ oraz $k \in M$.
\end{itemize}


\breakparagraph{}
Zakładamy przy tym, że moment rozpoczęcia procesu to:
\begin{equation}
	S_{1, \pi(1)}=0
\end{equation}

\breakparagraph{}
Zakończenie wszystkich zadań:
\begin{equation}
	C_{\max}(\pi) = 	C_{m, \pi(n)} = S_{m, \pi(n)} + p_{m, \pi(n)}
\end{equation}

\breakparagraph{}
W naszym przypadku optymalizacja procesu będzie polegała na wyznaczeniu optymalnej permutacji zadań $\pi^*$ takiej, że:
\begin{equation}
	C_{\max}(\pi^{*}) = \min\{C_{m, \pi{(n)}^{*}}\} \ \pi \in \Pi
\end{equation}
