\chapter{Sformułowanie problemu}
Problem poruszany w pracy to problem przepływowy (ang. Flow Shop) z przezbrojeniem maszyn.
W procesie jest wykorzystywane 6 stanowisk (maszyn) które składają się na cały proces.
W procesie na dwóch maszynach wykonuję sie przezbrojenie. Jest to stanowisko z sitodrukiem oraz pick\&place.

Zbiór maszyn
\begin{equation}
	M=\lbrace 1, 2, \dots, m \rbrace
\end{equation}
gdzie $m$ to liczba maszyn.

Zbiór zadań:
\begin{equation}
	J=\lbrace 1, 2, \dots, n \rbrace
\end{equation}
gdzie $n$ to liczba zadań.


Zadanie $j \in J$ jest ciągiem $m$ operacji:
\begin{equation}
	O_{1, j}, O_{2, j}, \dots, O_{m, j}
\end{equation}

Zmienne w modelu:
\begin{itemize}
	\item $p_{k, j}$ --- czas trwania operacji z zadania $j$ na maszynie $k$
	\item $S_{k, j}$ --- moment rozpoczęcia operacji z zadania $j$ na maszynie $k$
	\item $C_{k, j}$ --- moment zakończenia operacji z zadania $j$ na maszynie $k$
\end{itemize}

Czas przezbrojenia pomiędzy zadaniem $i$ oraz $j$ na $k$-tej maszynie to $s_{i, j}^{k}$ gdzie:
\begin{itemize}
	\item $i \neq j$
	\item  $i, j \in J$
\end{itemize}

Permutacja zadań $\pi = \lbrace \pi(1), \pi(1), \dots, \pi(n) \rbrace$ gdzie $n$ to liczba zadań.

Moment początkowy operacji:
\begin{equation}
	S_{k, \pi(j)}=\max\{C_{k-1, \pi(j)}, C_{k, \pi(j-1)}+s^k_{\pi(j-1), \pi(j)}\},
\end{equation}

Obliczenie momentu zakończenia operacji:
\begin{equation}
	C_{k, \pi(j)} = S_{k, \pi(j)} + p_{k, \pi(j)}
\end{equation}

Gdzie $ j \in J$ oraz $k \in M$.

Moment początkowy procesu:
\begin{equation}
	S_{1, \pi(1)}=0
\end{equation}

Moment końcowy procesu:
\begin{equation}
	C_{\max} = 	C_{m, \pi(n)} = S_{m, \pi(n)} + p_{m, \pi(n)}
\end{equation}

W naszym przypadku optymalizacja procesu będzie polegała na wyznaczeniu optymalnej permutacji zadań $\pi^*$
w celu znalezienia minimalnego czasu $C_{\max}$
\begin{equation}
	C_{\max}^{*} = \min\{C_{m, \pi{(n)}^{*}}\}
\end{equation}

Z racji że proces jest rzeczywisty to posiada pewne ograniczenia:
\begin{itemize}
	\item musi być zachowany porządek technologiczny operacji,
	      \begin{equation}
	      	O_{1, j} \rightarrow O_{2, j} \rightarrow \dots \rightarrow O_{m, j}
	      \end{equation}
	\item każda operacja może być wykonywana tylko przez jedną maszynę,
	\item żadna maszyna nie może wykonywać więcej jak jednej operacji (uwagi: przypadek kiedy w piecu lutowniczym lutujemy jedną płytę PCB)
	\item wykonywanie żadnej operacji nie może zostać przerwane przed zakończeniem,
	\item momenty rozpoczęcia operacji oraz czas przezbrojenia nie mogą być ujemne
	      \begin{equation}
	      	S_{k, \pi(i)}, s^{k}_{\pi(i), \pi(j)} > 0
	      \end{equation}
\end{itemize}
